\chapter{Conclusion}
\label{conclusion}

In this thesis, we have introduced five mechanisms to decentralize or disintermediate certain aspects of blockchain-based marketplaces.
Using our universal ConTrib accounting mechanism, market information can securely be stored in a decentralized manner by peers themselves.
With our decentralized MATCH middleware, participants can perform order matchmaking in peer-to-peer markets themselves.
Our XChange trading mechanism enables asset trading between permissioned blockchains without any requirement for a trusted third party.
The decentralized Internet-of-Money overlay enables fast and international money transfers without gross settlement by a central bank.
Finally, DevID and DAppCoder is capable of disintermediating software crowdsourcing markets and allows software developers to build self-hosted, durable portfolios without data management by a trusted third party.

\section{Conclusions}
The main conclusions of this thesis are as follows:

\begin{enumerate}
	\item In Chapter~\ref{chapter:trustchain}, we have built ConTrib, a universal accounting mechanism.
	With a two-year deployment trial of ConTrib in our peer-to-peer application Tribler, we have successfully addressed free-riding behaviour in our Tor-like overlay.
	Our ConTrib mechanism is highly suitable for accounting data within different application domains.
	In this thesis alone, we have leveraged ConTrib for accounting of data elements in XChange (Chapter~\ref{chapter:xchange}), Internet-of-Money (Chapter~\ref{chapter:iom}) and DevID (Chapter~\ref{chapter:devid}).
	
	\item Our decentralized MATCH middleware, presented in Chapter~\ref{chapter:match}, is resilient against manipulation during order matchmaking.
	This manipulation is a significant concern in peer-to-peer markets under central ownership.
	MATCH performs high-quality matchmaking and does so with bandwidth and memory overhead orders of magnitude lower compared to matchmaking on a blockchain.
	We are the first to experiment with a fair alternative to the Uber ride-hailing market.
	
	\item In Chapter~\ref{chapter:xchange}, we have presented a new approach for asset trading between permissioned blockchains.
	Compared to existing trading approaches that either require third party intervention or modifications to deployed blockchain logic, our approach is fully decentralized and is compatible with all permissioned blockchains.
	Peers account ll trading activity in a distributed log, and users will not trade with suspected fraudsters until an identified dispute is resolved.
	This approach limits the economic damage that adversaries can cause.
	
	\item In Chapter~\ref{chapter:iom}, we show how we have reduced the settlement duration of intra-bank payments from days to mere seconds.
	Our decentralized overlay, Internet-of-Money, is fully compatible with existing banking infrastructure.
	
	\item In Chapter~\ref{chapter:devid}, we introduce unified, blockchain-based portfolios with DevID.
	DevID solves the problem that the portfolio of developers is usually scattered across centralized platforms, and vendor locked, making it hard to get an accurate impression of the developers' skills.
	Our fully decentralized software crowdsourcing leverages DevID portfolio to match clients with developers, without trusted intermediaries.
	
	\item \emph{Detecting} fraud, instead of preventing it, is an efficient and overlooked approach that can be applied to improve different aspects of blockchain-based marketplaces.
	In Chapter~\ref{chapter:trustchain}, we have demonstrated that fraud, targeted at the ConTrib data structure, can be detected within seconds.
	In Chapter~\ref{chapter:xchange}, we detect fraud and violate the liveness of malicious peers, preventing them from causing further harm.
	Finally, in Chapter~\ref{chapter:iom}, we leverage fraud detection to identify malicious money routers and show that the economic gains by adversaries are manageable.
	
	\item Incremental settlement, the act of breaking up an individual payment into multiple smaller ones, is an effective risk mitigation strategy.
	We have applied this strategy to reduce value-at-stake in our XChange trading mechanism (Chapter~\ref{chapter:xchange}) and our Internet-of-Money overlay (Chapter~\ref{chapter:iom}).
\end{enumerate}

% Full fraud prevention is often unnecessary

% Incremental payments are awesome

\section{Future Directions}
Many opportunities remain to decentralize and disintermediate the aspects of blockchain-based marketplaces.
We end this thesis by outlining a few ideas per chapter for further directions.

\begin{enumerate}
	\item in Chapter~\ref{chapter:trustchain}, we have introduced the universal accounting mechanism ConTrib.
	Interesting future work would be to improve the probability of fraud detection further when sharing records.
	We believe that more sophisticated dissemination techniques can improve the security of our mechanism.
	For example, the dissemination of records can take the record payload into consideration, where \enquote{important} records are shared amongst more peers.
	Another research angle can focus on an adaptive number of back-pointers in records, where \enquote{important} records bear more links.
	
	\item In Chapter~\ref{chapter:match}, we have presented MATCH, our decentralized middleware for fair matchmaking in peer-to-peer markets.
	Even though we show that MATCH is highly resistant against malicious matchmakers, we considered the identification of such matchmakers outside the scope of our work.
	A natural extension of MATCH would be to leverage ConTrib and account order dissemination and proposed matches.
	By replaying the events in ones personal ledger, deviation from a particular matching policy can be detected.
	We also suggest exploring statistical approaches for the detection of malicious behaviour.
	Specifically, our random dissemination model results in a particular distribution of the order book entries over peers in the network.
	By inspecting incoming match proposals, long-term deviation from a matching policy can then be detected with a certain probability.
	
	\item in Chapter~\ref{chapter:xchange}, we have introduced a universal asset trading mechanism between permissioned blockchains.
	An important question that remains is how our mechanism can be used to trade assets between public blockchains, for example, Ethereum.
	The critical problem when deploying our mechanism in a public setting is the ability to quickly generate a new identity after committing fraud.
	The use of collateral deposits that are slashed when a peer undisputedly commits fraud can help to avoid this threat.
	
	\item In Chapter~\ref{chapter:iom}, we have presented our international money transfer mechanism, named Internet-of-Money.
	A shortcoming of our approach is that the magnitude of an intra-bank payment is limited by the balance constraints of money routers in the circuit.
	Once a money router has depleted its balance in one connected bank account, this router might be unable to route further payments.
	Rebalancing the router requires a conventional payment which can be slow.
	A promising research avenue is to rebalance money routers using Internet-of-Money functionality itself.
	
	\item In Chapter~\ref{chapter:devid}, we have introduced blockchain-based portfolios for software developers.
	We also designed and implemented a decentralized crowdsourcing market.
	We believe that there are opportunities to devise new processes for securely linking third-party assets with DevID portfolios.
	Possible research efforts can focus on large-scale deployment of DAppCoder and integration of our tools in collaboration software like GitHub.
	This would also include a cryptocurrency-based remuneration system.
\end{enumerate}