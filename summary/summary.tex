\chapter*{Summary}
\addcontentsline{toc}{chapter}{Summary}
\setheader{Summary}

Digital marketplaces play an essential role in our economy.
The standard approach to devise such marketplaces is by deploying centralized infrastructure, entirely operated and managed by an authoritative market operator.
Traditional marketplaces often use trusted intermediaries to store the available market information and to handle different parts of a trade, such as payment processing and dispute resolution.

Advancements in information technology have challenged the need for both authoritative market operators and trusted intermediaries.
These advancements have resulted in blockchain-based marketplaces that are capable of facilitating trade directly between peers, without centralized control and trusted intermediaries.
These marketplaces adopt blockchain technology to carry out one or more of its critical operations by replacing social trust with cryptographic primitives.
As such, blockchain-based marketplaces are capable of \emph{decentralizing} and \emph{disintermediating} different components of digital marketplaces.
Decentralization refers to the concept of delegating decision making and activities away from a central authority.
Disintermediation reduces or removes the involvement of trusted intermediaries when trading.

This thesis introduces innovative approaches to decentralize and disintermediate all aspects of blockchain-based marketplaces.
We identify five aspects of blockchain-based marketplaces, namely information management, matchmaking, settlement, fraud management, and identity management.
We then design, implement and deploy five decentralized mechanisms.
Each introduced mechanism focusses on one or two aspects of blockchain-based marketplaces.
Throughout this thesis, we adopt an experimental approach where feasibility and real-world deployment are essential goals of our proposed solutions.

In Chapter 1, we describe the five identified aspects of blockchain-based marketplaces and outline existing approaches that disintermediate and decentralize these aspects.
We present our research questions, outline our research and engineering methodology, and summarize the key contributions of this work.

In Chapter 2, we introduce a universal accounting mechanism, named ConTrib, to account work in decentralized applications.
With ConTrib, each peer maintains a \emph{personal ledger} containing tamper-evident records.
A record describes performed work between peers and links to other records.
Fraud, operating multiple copies of a personal ledger in secret, is detected by continuously exchanging records and by verifying the consistency of incoming records against known ones.
We perform a two-year deployment trial of ConTrib to address free-riding behaviour in Tribler, our decentralized file-sharing application.
ConTrib also underlines several other mechanisms introduce in this thesis.

In Chapter 3, we introduce MATCH, a decentralized middleware for fair matchmaking in peer-to-peer markets.
MATCH addresses manipulation concerns associated with marketplaces under central control.
By decoupling the dissemination of potential matches from the negotiation of trade agreements, MATCH empowers end-users to make their own educated decisions and to engage in direct negotiations with trade partners.
This approach makes MATCH highly resilient against malicious matchmakers that deviate from a specific matching policy.

In Chapter 4, we introduce a universal and decentralized settlement mechanism, named XChange.
XChange enables trustworthy asset trade between permissioned blockchains without the requirement for a trusted intermediary, collateral deposits or modifications to deployed applications.
Our mechanism records the progression of each trade within records on a distributed log.
XChange limits the economic gains of adversaries that have committed fraud during a trade by preventing them from engaging in other trades.
By inspecting the trade records in the distributed log, every participant can detect if a trader is refraining from fulfilling its obligations during an ongoing trade.

In Chapter 5, we introduce Internet-of-Money, a mechanism for real-time and international money transfers between different banks.
The key idea is to break up a slow intra-bank payment into multiple fast inter-bank payments.
A money transfer uses one or more volunteer-based services, \emph{money routers}, to complete an intra-bank payment.
This approach reduces the duration of intra-bank payments from days to mere seconds.
To identity fraud by money routers, we record all payments with our ConTrib accounting mechanism.
We build and deploy a decentralized overlay network, compatible with existing banking infrastructure, to discover available money routers and to send money to arbitrary bank accounts.

Finally, in Chapter 6, we introduce DevID, an identity management solution for software developers.
DevID bundles developer information within records on a distributed ledger.
Developers can import data assets from third parties into a unified DevID portfolio, add projects and skills, and receive endorsements.
To demonstrate the practical value of DevID, we build and deploy a decentralized crowdsourcing marketplace that utilizes DevID portfolios.
Our user trial demonstrates that DevID is efficient at storing portfolio records.

\chapter*{Samenvatting}
\addcontentsline{toc}{chapter}{Samenvatting}
\setheader{Samenvatting}

{\selectlanguage{dutch}

TODO
}



