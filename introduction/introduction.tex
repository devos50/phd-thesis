\chapter{Introduction}
\label{introduction}

%\newpage

\dropcap{E}ver since the introduction of the Internet, individuals all over the world rely on it to interact and exchange value.
Nowadays, major companies have built large-scale digital platforms to facilitate these on-line interactions between potential strangers.
For example, eBay was one of the first platforms that enable trustworthy trade of goods between merchants and buyers and is used by millions on a daily basis.
Major companies acting in the sharing economy, like Uber and AirBnb, bootstrapped an ecosystem where strangers share resources like their house or cars.

Most of the popular online services are facilitated by centralized architectures, often deployed and maintained by a single company.
The main characteristic of a centralized architecture is that there is a single, or limited group of servers, responsible for handling all requests submitted by platform participants.
Hosting the network infrastructure to facilitate interactions on a global scale requires major investments and platform costs, as exemplified by major companies like Facebook and Google.

On the other hand, decentralized architectures aim to avoid dependencies on centralized entities and leverage the computing power of the participants instead.
One of the most popular decentralized applications is the BitTorrent file transfer protocol.
In BitTorrent, users directly exchange parts a (potentially large) file with each other, without involvement of any centralized server.
Although decentralized applications are often harder to shut down, they often are more vulnerable to targeted attacks, such as the Sybil Attack and Eclipse Attack.

The introduction of the decentralized Bitcoin currency in 2008 by Satoshi Nakamoto\footnote{A pseudonym, the real identity is currently (still) unknown.} bootstrapped a new era of decentralized applications.
Bitcoin is the first popular currency system not under control by any financial institution and gained much interest from both academia and industry.
The system is powered by blockchain technology, a tamper-proof distributed ledger maintained by all network participants.
Currently, blockchain technology is being applied to solve problems within a large number of domains, including finance, health care, identity and real estate.

Despite much excitement and potential applications, existing blockchain-based solutions are by far not powerful enough to be applied at a global scale.
Specifically, the rate at which new transactions can be appended to the blockchain is often limited due to the requirement to reach consensus on the validity of included transactions.
For example, the throughput of Bitcoin is theoretically bounded by seven transactions per second.
In comparison, credit companies like VISA are capable of processing thousands of transactions per second.

% applications are based on distributed ledger technology, where 

In this thesis, we focus on ...

\section{Distributed Ledger Technology}
TODO

\section{Blockchain Technology}
TODO

\subsection{Cryptocurrencies and Digital Tokens}
TODO

\subsection{Smart Contracts}
TODO

\subsection{The Quest for Scalability}
TODO

%\section{Problem Statement}
%The key issue is that decentralized applications that are using blockchain technology are often not scalable enough.

\section{Research Questions}

The overarching research question of this thesis is as follows:

\emph{how can we increase the scalability of decentralized applications that are using blockchain technology?}\\\\
In order to answer our research question, we address the following key questions:

\textbf{[RQ1] What are the practical scalability limitations of state-of-the-art blockchains?}
Since the inception of the Bitcoin cryptocurrency, there has been much effort to improve the throughput (scalability) of blockchains in general.
This has resulted in a large range of different design proposals for blockchain ledgers, each with their own trade-offs.
Although there has been much research around the theoretical performance of different blockchain fabrics, the field lacks a rigorous evaluation from a practical perspective.

\textbf{[RQ2] How can we design and implement a scalable and secure blockchain, based on fraud detection?}
Most existing blockchains are designed around fraud prevention, preventing invalid transactions from entering the distributed ledger.
This guarantee limits the rate at which new transactions can be appended to the distributed ledger, therefore limiting scalability.
Designing a blockchain ledger around fraud detection instead could improve scalability and lead to new types of decentralized applications.
% reactive approach vs proactive

\textbf{[RQ3] How can we apply accountability and fraud detection to improve the speed of (international) payment systems?}
Similar to RQ2, existing payment systems are built to prevent fraud and often highly complex in nature.
As a result, a payment between two different (international) banks often takes a long time to settle, often in the order of days, and incurs significant transaction fees.
Relying on fraud detection could potentially improve the speed of (international) payment systems, since the security requirements are different.

\textbf{[RQ4] How can we apply accountability and fraud detection to improve decentralized exchanges?}

\textbf{[RQ5] How can we apply accountability and fraud detection to build scalable portfolios for developers?}

\section{Research Methodology}

% add: related research mostly follows a theory-driven approach

Throughout this thesis, we adopt an experimental research methodology to answer our research questions.
We validate our ideas by implementing them in software and evaluating them with emulation-based experimentations on our DAS5 compute cluster.
While designing an infrastructure or system, we always aim for a solution that is applicable in the real world.

We believe this experimental research methodology is suitable for two reasons.
First, it allows us to evaluate our ideas in an environment that closely resembles a real-world environment.
Second, it directly leads to a software implementation that can be used by academia or industry.
All developed software artefacts are available on our GitHub repository and contain unit tests to verify their functional correctness.

% existing research does X

% but we do Y!

\section{Contributions and Thesis Outline}

After establishing the required background on decentralized applications and distributed ledger technology, Chapter \todo{X} highlights scalability limitations of state-of-the-art blockchain ledgers.
Next, in Chapter \todo{X}, we design, implement and evaluate TrustChain: a scalable blockchain ledger that is based on fraud \emph{detection} instead of fraud \emph{prevention}.
In Chapters \todo{X to Y}, we apply accountability primitives and fraud detection techniques in three well-established domains: financial transactions, two-sided marketplaces, and identity.
Specifically, we make the following contributions in each chapter:\\

\textbf{[Chapter 1] Scalability Limitations of Existing Blockchains.} (pending, not sure whether this will be a separate chapter)
Some explanation here...\\

\textbf{[Chapter 2] TrustChain: A Scalable Blockchain for Accounting.}
In this chapter, we answer RQ2 by introducing TrustChain, a scalable blockchain built for accounting of primitive data elements.
Our permission-less fabric is capable of creating trusted transactions amongst strangers without any central oversight or requirement for network-wide consensus.
In contrast to existing blockchain ledgers, TrustChain is designed to \emph{detect} fraud, rather than \emph{prevent} fraud.
The key design principle is that each individual grows and maintains their own ledger with historical transactions.
When initiating a transaction with another user, their ledgers become entangled.
We highlight the maturity and applicability of our distributed ledger by integrating TrustChain in the Tribler peer-to-peer file-sharing application to account bandwidth exchange between individuals.
Using the tamper-proof transactions stored on TrustChain, we design a fair load-balancing mechanism used by participants in the anonymous downloading overlay and show how to ensure preferential treatment for well-behaviour users in the network. % free-riding behaviour is addressed.
Our real-world deployment and evaluation has resulted in \todo{X} TrustChain transactions, created by \todo{X} unique digital identities.
This chapter is largely based on the following publication:

Pim Otte, Martijn de Vos and Johan Pouwelse, \enquote{TrustChain: A Sybil-resistant scalable blockchain}, \emph{Future Generation Computer Systems (FGCS), 2017.}\\

\textbf{[Chapter 3] Internet-of-Money: Applying Accountability to Enable Real-time International Money Transfers.}
In this chapter, we answer RQ3 and explore a new stage in the evolution of digital trust, trusting strangers with your money.
We address the challenging problem of giving money to others and relying on them to forward it.
To identity fraud, we account money transfers between interacting strangers.
This work represents a small step towards a generic infrastructure for trust, moving beyond proven, single-vendor platforms like eBay, Uber and AirBnb.
Expanding upon trust relations, we designed, implemented and evaluated an overlay network: \emph{Internet-of-Money}.
Internet-of-Money is capable of real-time money transfers to different banks by routing funds through individuals (\emph{money routers}).
This removes the need for central banks to handle a payment.
Our network reduces traditional payment durations from a day or even a few days in weekends, to mere seconds.
%Internet-of-Money is fully decentralized, privacy-preserving and highly scalable.
With real-world experimentations, we prove that Internet-of-Money enables fast money forwarding.
We show that our overlay network is capable of discovering a majority of available money routers within a minute.
Finally, we demonstrate how profit of cheating routers is limited and that misbehaviour is punished.
This chapter is based on the following publication:

Martijn de Vos and Johan Pouwelse, \enquote{Real-time Money Routing by Trusting Strangers with your Funds}, \emph{IFIP Networking, 2018.}\\

\textbf{[Chapter 4] MATCH: Accountable and Generic Matchmaking for Decentralized Applications}
In this chapter, we partially address RQ4 and focus on matchmaking, the process of bringing market participants together based on individual preferences.
Matchmaking is a cardinal, yet overlooked prerequisite for a fully decentralized exchange.
Although numerous companies have deployed infrastructure for matchmaking, there is currently no solution that can be deployed within different trading domains.
We present MATCH, a middleware for generic order matching.
Since MATCH is agnostic about order specifications, the resulting system is highly flexible and reusable.
In this work, we first present an alternative approach to order matching, named \emph{decentralized matchmaking}.
The main idea is that new orders are disseminated to multiple matchmakers simultaneously and shared between matchmakers.
We then design and implement a novel matching protocol and a middleware with full support for both existing matchmaking paradigms and decentralized matchmaking.
Finally, we extensively evaluate desired system properties of MATCH under a real-world ride-hailing and asset trading workload.
Our main finding is that decentralized matchmaking exhibits superior fault tolerance and load balancing, at the cost of moderately increased bandwidth usage and order completion time.
This chapter is based on the following publication:

\todo{UNDER REVIEW}\\

\textbf{[Chapter 5] XChange: A Scalable Asset Marketplace based on Accountability}\\

\textbf{[Chapter 6] DevID: Blockchain-based Portfolios for Software Developers}\\

\textbf{[Chapter 7] Conclusion.} We will end this thesis with the conclusions, a summary of the lessons learned, and suggestions for further work.